\documentclass[12pt,a4paper]{article}
\usepackage[utf8]{inputenc}
\usepackage[T1]{fontenc}
\usepackage[english]{babel}
\usepackage{csquotes} 
\usepackage[]{mcode}
\usepackage[left=0.3cm,right=0.7cm,top=1.6cm,bottom=1.6cm]{geometry}
\geometry{bindingoffset=1.5cm}
\usepackage{graphicx}
\usepackage[section]{placeins}
\usepackage[parfill]{parskip}
\usepackage{fancyhdr}
\setlength{\headheight}{15.2pt}
\pagestyle{fancy}
\lhead{}
\chead{\textit{ \nouppercase{\leftmark}}}
\rhead{}
\usepackage{hyperref}

\begin{document}	
	
	\title{Study of the influence of educational factors on diplomas rate through PISA results}
	\author{Simon \textsc{Lebastard} \and Cyrille \textsc{Vessaire}}
	\date{February, 14th, 2016}
	
\maketitle
	
\section{Motivation}

\subsection{General approach}

The aim of this statistical study is to unveil the influence of some measurable factors on the performances of education in the countries of OECD.

The authors already have a few years of experience in giving classes to secondary level students, and are interested in the influence of the context on learning performances. Quite often have they heard their students complain about an uninteresting mathematics teacher giving class facing a blackboard, turning is back on a crowded and turbulent 35 students classroom. The authors already had the opportunity to try out and evaluate the efficacity of several teaching methods when teaching to one to six students. However they have no experience in high-effective teaching, which is at the same time the most widespread mean of teaching in public and private institutions, and almost the most challenging way to share knowledge.
\\
Some factors are believed to have influence on the learning proficiency. Among others, the following factors were chosen for study:

\begin{itemize}
	\item The student to teaching staff ratio in a given school
	\item The number students in a given class
	\item The teacher's salaries
	\item The amount of teaching hours
	\item The amount of homework as measured in preperation hours
	\item The spending on education
	\item Teaching methods that will be referred to as \textit{innovative methods} in this paper:
		\subitem the amount of independent work in class
		\subitem the valorisation of explanations and ideas over mere results
		\subitem working in small groups of students
		\subitem giving the student the choice of the method to chose over the resolution of non-trivial problems
		\subitem forming classes by abilities in a given field 
\end{itemize}

There are great economic and cultural discrepencies in the world, and it appeared complicated to the authors to study the effects of those factors on a large panel of countries with completely different situations. What the authors aim at instead is to study their impact for a collection of countries of comparable wealth.
Therefore the OECD database was chosen over some other available source of data, and the paper will only deal with the corresponding countries.

\subsection{Chosen factors}

\subsubsection{Number of student in a class \& student to teaching staff ratio}

One of the arguments to the uneffectivemess of teaching in many secondary institutions is the high amount of students in the same class, causing crowded and noisy classes, which can be highly detrimental for focusing.

The students to teachers ratio does not measure this density, but reflects the way students are framed during their learning hours.

\subsubsection{Teachers salaries}

This is a key factor, as teaching salaries settle the attractiveness of the job, and thus the competition among teachers. At a given level of demand in teachers, high salaries will lead to better selection among teachers. One could then expect better paid teachers to be more skilled. This study will try to assess whether this is true or not, and if it is. Well, not really, as what is measured is only the results of the students at standard tests, and it is obvious that a skilled mathematician may not be a good teacher. There are two things that will be measured regardless of one another: the skills of the teachers, and the way teachers are selected.

\subsubsection{Time spent in class \& at doing homework}

This study will try to see if there is a relation between the amount of time students spentd in time and their academic results. This is interesting namely because even among western european countries, class hours and academic calendars are very different. Here the only factor studied is the number of hours spent in class per week. Holidays are therefore not taken into account.

\subsection{Measurements of effectiveness}

Relate PISA to third grade enrollment and diplomas rate

\subsubsection{Link to work}



\section{Dependency of PISA results in different fields}



\end{document}